\newgeometry{left=2cm, top=1cm, right=2cm, bottom=1cm}

\noindent\begin{minipage}[t]{0.43\textwidth}
    \centering
    \fontsize{11pt}{16.5pt}
    \textbf{Trường ĐH Sư Phạm Kỹ Thuật TP.HCM}\\
    \textbf{Khoa CNTT}\\
    % \rule[12pt]{0.5\textwidth}{.4pt}
    *******
\end{minipage}

\begin{center}
    \fontsize{18pt}{27pt}
    \textbf{ĐỀ CƯƠNG LUẬN VĂN TỐT NGHIỆP}
\end{center}

\begin{table}[!h]
    \centering
    \begin{tabularx}{0.8\textwidth}{ X c }
        Họ và Tên SV thực hiện 1: Huỳnh Minh Phước   & Mã Số SV: 20133082 \\
        Họ và Tên SV thực hiện 2: Văn Mai Thanh Nhật & Mã Số SV: 20133076 \\
        Họ và Tên SV thực hiện 3: Trần Đông          & Mã Số SV: 20133035 \\
    \end{tabularx}
\end{table}

\begin{center}
    \begin{tabular}{p{0.35\textwidth} p{0.45\textwidth}}
        \multicolumn{2}{l}{Thời gian làm luận văn từ\hspace{3cm} đến} \\
        Chuyên ngành: & Kỹ thuật dữ liệu                              \\
        Tên luận văn: & Nhận diện văn bản tiêu cực sử dụng học máy    \\
        GV hướng dẫn: & TS. Trần Nhật Quang
    \end{tabular}
\end{center}

\textbf{Nhiệm Vụ Của Luận Văn :}
\begin{enumerate}
    \item Tìm hiểu các kiến trúc mạng neuron và các thuật toán dùng trong xử lý văn bản.
    \item Tìm hiểu về xử lí ngôn ngữ tự nhiên.
    \item Tìm hiểu các kỹ thuật nhúng từ (word embedding).
    \item Tìm hiểu các thư viện, module hỗ trợ học máy và học sâu như Tensorflow, Sklearn, \dots
    \item Tìm hiểu những kiến trúc học sâu như LSTM, RNN, Transformer, \dots và các mô hình tiền huấn luyện GPT, BERT, \dots
    \item Ứng dụng các kiến thức đã tìm hiểu vào việc xây dựng mô hình phát hiện từ ngữ độc hại.
    \item Áp dụng mô hình đã xây dựng vào những ứng dụng thực tiễn.
\end{enumerate}

\text{Đề cương viết luận văn:}

\begin{center}
    \textbf{MỤC LỤC}
\end{center}
\begin{enumerate}[label=\textbf{\arabic*.}]
    \item \textbf{Phần MỞ ĐẦU}
    \item \textbf{Phần NỘI DUNG: Gồm có 4 chương}
          \begin{enumerate}
              \item Chương 1: Tổng quan về học máy và học sâu
                    \begin{itemize}
                        \itemsep 0pt
                        \item Khái quát về học máy
                        \item Khái quát về học sâu
                    \end{itemize}
              \item Chương 2: Mạng neuron nhân tạo
                    \begin{itemize}
                        \item Mạng neuron nhân tạo
                        \item Mạng neuron tích chập - CNN (Convolutional Neural Network)
                        \item Mạng neuron hồi quy - RNN (Recurrent Neural Network)
                        \item Bộ nhớ dài-ngắn hạn - LSTM (Long Short-Term Memory)
                        \item Bộ nhớ tái phát - GRU (Gated Recurent Unit)
                    \end{itemize}
              \item Chương 3: Xử lý ngôn ngữ tự nhiên
                    \begin{itemize}
                        \item Xử lý ngôn ngữ tự nhiên
                        \item Kỹ thuật nhúng từ (Word embedding)
                        \item Ngữ cảnh (Contextual) và vai trò trong NLP
                        \item Mô hình Transformer
                        \item Tiếp cận nông và học sâu trong ứng dụng pre-training NLP
                        \item Mô hình BERT (Bidirectional Encoder Representations from Transformers)
                    \end{itemize}
              \item Chương 4: Xây dựng mô hình phát hiện từ ngữ độc hại
                    \begin{itemize}
                        \item Môi trường cài đặt và các thư viện sử dụng
                        \item Mô tả tập dữ liệu
                        \item Tiền xử lý dữ liệu
                        \item Thiết lập mô hình
                        \item Huấn luyện mô hình và đánh giá kết quả
                    \end{itemize}
              \item Chương 5: Ứng dụng mô hình vào thực tiễn
                    \begin{itemize}
                        \item Giới thiệu
                        \item Mục đích
                        \item Quá trình phát triển
                        \item Cách hoạt động
                    \end{itemize}
          \end{enumerate}
    \item \textbf{Phần KẾT LUẬN}
    \item Tài liệu tham khảo
\end{enumerate}

% plan
\newpage
\textbf{KẾ HOẠCH THỰC HIỆN}
% cái này của tiểu luận
% \thispagestyle{empty}
% \begin{table}[h]
%     \centering
%     \begin{tabular}{
%         |>{\centering\arraybackslash}p{0.1\textwidth}
%         |>{\centering\arraybackslash}p{0.2\textwidth}
%         |>{\arraybackslash}p{0.5\textwidth}|
%         }
%         \hline
%         \textbf{Tuần} & \textbf{Thời gian} & \textbf{Nội dung công việc}                                      \\\hline
%         27            & 11/9 - 17/9        & Tìm hiểu sơ lược về đề tài.                                      \\\hline
%         28            & 18/9 - 24/9        & Tìm hiểu học máy và học sâu.                                     \\\hline
%         29            & 25/9 - 8/10        & Tìm hiểu mạng thần kinh truyền thống.                            \\\hline
%         30            & 9/10 - 15/10       & Tìm hiểu mạng thần kinh tích chập.                               \\\hline
%         31            & 16/10 - 22/10      & Tìm hiểu mạng thần kinh hồi quy.                                 \\\hline
%         32            & 23/10 - 29/10      & Tìm hiểu bộ nhớ dài-ngắn hạn và bộ nhớ tái phát.                 \\\hline
%         33            & 30/10 - 5/11       & Tìm hiểu kỹ thuật nhúng từ, mô hình Word2vec, GloVe và Fasttext. \\\hline
%         34            & 6/11 - 12/11       & Tìm hiểu cách triển khai đề tài, tìm và mô tả tập dữ liệu.       \\\hline
%         35            & 13/11 - 19/11      & Tiền xử lý dữ liệu.                                              \\\hline
%         36            & 20/11 - 17/12      & Thiết lập và huấn luyện mô hình, đánh giá kết quả.               \\\hline
%         37            & 18/11 - 23/12      & Hoàn thiện báo cáo.                                              \\\hline
%     \end{tabular}
% \end{table}

% sửa lại một chút từ tiểu luận
% \begin{table}[h]
%     \centering
%     \begin{tabular}{
%         |>{\centering\arraybackslash}p{0.1\textwidth}
%         |>{\centering\arraybackslash}p{0.2\textwidth}
%         |>{\arraybackslash}p{0.5\textwidth}
%         |>{\arraybackslash}p{0.15\textwidth}|
%         }
%         \hline
%         \textbf{STT} & \textbf{Thời gian} & \textbf{Công việc} & \textbf{Ghi chú}                                      \\\hline
%         1            & 19/02 - 24/02      & Tìm hiểu sơ lược về đề tài.                                      &     \\\hline
%         2            & 26/02 - 02/03      & Tìm hiểu học máy và học sâu.                                     &     \\\hline
%         3            & 04/03 - 09/03      & Tìm hiểu về mảng xử lý ngôn ngữ tự nhiên.                        &     \\\hline
%         4            & 11/03 - 16/03      & Tìm hiểu mạng thần kinh truyền thống.                            &     \\\hline
%         5            & 18/03 - 23/03      & Tìm hiểu mạng thần kinh tích chập.                               &     \\\hline
%         6            & 25/03 - 30/03      & Tìm hiểu mạng thần kinh hồi quy.                                 &     \\\hline
%         7            & 01/04 - 06/04      & Tìm hiểu bộ nhớ dài-ngắn hạn và bộ nhớ tái phát.                 &     \\\hline
%         8            & 08/04 - 13/04      & Tìm hiểu kỹ thuật nhúng từ, mô hình Word2vec, GloVe và Fasttext. &     \\\hline
%         9            & 15/04 - 27/04      & Tìm hiểu kiến trúc mô hình Tranformer, cơ chế Attention          &     \\\hline
%         10           & 29/04 - 11/05      & Tìm hiểu kiến trúc mô hình BERT, PhoBert                         &     \\\hline
%         11           & 13/05 - 18/05      & Tìm hiểu cách triển khai đề tài, tìm và mô tả tập dữ liệu.       &     \\\hline
%         12           & 20/05 - 25/04      & Tiền xử lý dữ liệu.                                              &     \\\hline
%         13           & 27/05 - 29/06      & Thiết lập và huấn luyện mô hình, đánh giá kết quả.               &     \\\hline
%         14           & 01/07 - 06/07      & Hoàn thiện báo cáo.                                              &     \\\hline
%     \end{tabular}
% \end{table}

% mới
\begin{table}[h]
    \centering
    \begin{tabular}{
        |>{\centering\arraybackslash}p{0.05\textwidth}
        |>{\centering\arraybackslash}p{0.15\textwidth}
        |>{\arraybackslash}p{0.5\textwidth}
        |>{\arraybackslash}p{0.15\textwidth}|
        }
        \hline
        \textbf{STT} & \textbf{Thời gian} & \textbf{Công việc}                                                            & \textbf{Ghi chú} \\\hline
        1            & 19/02 - 24/02      & Tìm hiểu bộ nhớ dài-ngắn hạn và bộ nhớ tái phát.                              &                  \\\hline
        2            & 26/02 - 02/03      & Tìm hiểu kỹ thuật nhúng từ, mô hình Word2vec, GloVe và Fasttext.              &                  \\\hline
        3            & 04/03 - 09/03      & Tìm hiểu về mảng xử lý ngôn ngữ tự nhiên, cơ chế Fine-tuning                  &                  \\\hline
        3            & 11/03 - 16/03      & Tìm hiểu kiến trúc mô hình Tranformer, cơ chế Attention                       &                  \\\hline
        4            & 18/03 - 23/03      & Tìm hiểu kiến trúc mô hình BERT, PhoBert                                      &                  \\\hline
        5            & 25/03 - 30/03      & Tìm hiểu cách triển khai đề tài, tìm hiểu và mô tả tập dữ liệu.               &                  \\\hline
        6            & 01/04 - 13/04      & Tiền xử lý dữ liệu.                                                           &                  \\\hline
        7            & 15/04 - 11/05      & Xử lý dữ liệu, thiết lập và huấn luyện mô hình LSTM và GRU, đánh giá kết quả. &                  \\\hline
        8            & 13/05 - 08/06      & Xử lý dữ liệu, thiết lập và tinh chỉnh mô hình PhoBert, đánh giá kết quả.     &                  \\\hline
        9            & 10/06 - 15/06      & So sánh, đánh giá hiệu quả thực nghiệm các mô hình.                           &                  \\\hline
        10           & 17/07 - 22/06      & Xây dựng Tiện ích nhận diện cho trình duyệt.                                  &                  \\\hline
        11           & 24/06 - 29/06      & Xây dựng chatbot Discord.                                                     &                  \\\hline
        12           & 01/07 - 06/07      & Hoàn thiện báo cáo.                                                           &                  \\\hline
    \end{tabular}
\end{table}

\noindent\begin{minipage}[t]{0.42\textwidth}
    \centering
    \textbf{Ý kiến của giáo viên hướng dẫn}
    % \rule[12pt]{0.5\textwidth}{.4pt}
\end{minipage}\hfil
\begin{minipage}[t]{0.57\textwidth}
    \centering
    \textit{Tp. Hồ Chí Minh, ngày\qquad tháng\qquad năm 2024}

    \textbf{Người viết đề cương}
    % \rule[12pt]{0.5\textwidth}{.4pt}
\end{minipage}
\restoregeometry