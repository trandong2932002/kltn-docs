\subsubsection{Phương pháp đánh giá}
Để đánh giá tính chính xác của mô hình, ba phương pháp đánh giá được sử dụng là Precision, Recall và Accuracy. Với TP là True Positive, FP là False Positive, TN là True Negative, FN là False Negative.

Precision là tỉ lệ đối tượng được gán các kết quả giống nhau có kết quả thực sự giống nhau.
\begin{align}
    Precision=\dfrac{TP}{TP+FP}
\end{align}

Recall là tỉ lệ các đối tượng giống nhau được gán các kết quả giống nhau.
\begin{align}
    Recall=\dfrac{TP}{TP+FN}
\end{align}

Accuracy là tỉ lệ các đối tượng được gán đúng với kết quả mẫu.
\begin{align}
    Accuracy=\dfrac{TP+TN}{TP+FP+TN+FN}
\end{align}

\subsubsection{Tokenization}
Tokenization đánh số cho mỗi từ có trong tập từ. Dưới đây là minh họa sau khi thực hiện tokenize
tập dữ liệu.
\begin{table}[htb]
    \centering
    \caption{Các từ tương ứng sau tokenization}
    \begin{tabularx}{0.95\textwidth}{C{1}|C{1}|C{1}|C{1}|C{1}|C{1}|C{1}|C{1}|C{1}|C{1}}
        \toprule
        tôi & của & một & bạn & là & có & không & và & các & đã \\\midrule
        1   & 2   & 3   & 4   & 5  & 6  & 7     & 8  & 9   & 10 \\
        \bottomrule
    \end{tabularx}
\end{table}

\subsubsection{Word Embedding}
Mô hình nhóm sử dụng là mô hình fasttext. Do mô hình này hoạt động dựa trên cách tách các từ thành nhiều phần nhỏ, sẽ dễ dàng hơn để liên hệ với các từ viết thiếu rõ ràng. Mô hình fasttext được sử dụng là một mô hình đã được huấn luyện từ trước chuyên dụng cho tiếng Việt. Với 200000 là số từ tối đa có thể xuất hiện và 300 là số chiều của một vector từ, khởi tạo một ma trận với kích thước 200000\texttimes 300 để chứa các vector từ. Với mỗi từ đã được mô hình hóa bởi fasttext, lấy ra từ đó và index của chúng, sau đó lấy ra vector từ tương ứng. Sau cùng thay thế vector đã lấy vào vị trí tương ứng trong ma trận. Với cách làm này, với mỗi từ sau khi đã trải qua quá trình tokenization sẽ được nhận biết bằng vector nhúng từ tương ứng. (Ví dụ về việc sử dụng mô hình nhúng từ trong việc tìm các từ gần nghĩa \ref{table:fasttext-mostsimilar})