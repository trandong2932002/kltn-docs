Quá trình huấn luyện được thực hiện với ba mô hình khác nhau: hai mô hình dùng kiến trúc LSTM và GRU được huấn luyện từ đầu, một mô hình bằng cách tinh chỉnh mô hình PhoBERT của VinAIResearch.

\subsubsection{Mô hình LSTM}
Với mô hình LSTM có epochs = 32, batch = 128, mô hình gồm 4 lớp (bảng \ref{table:lstm-model}):
\begin{itemize}
    \item 1 lớp embedding sử dụng fasttext embedding
    \item 2 lớp Bidirectional LSTM
    \item 1 lớp dense
\end{itemize}

\begin{table}[htb]
    \centering
    \caption{Bảng tóm tắt mô hình LSTM}\label{table:lstm-model}
    \begin{tabular}{l l r}
        \toprule
        \textbf{Layer (type)}            & \textbf{Output Shape}                    & \textbf{Param \#} \\\midrule
        input (InputLayer)               & [(None, 200)]                            & 0                 \\
        embedding (Embedding)            & (None, 200, 300)                         & 60000000          \\
        bidirectional (Bidirectional)    & (None, 200, 128)                         & 186880            \\
        dropout (Dropout)                & (None, 200, 128)                         & 0                 \\
        bidirectional\_1 (Bidirectional) & (None, 128)                              & 98816             \\
        dense (Dense)                    & (None, 6)                                & 774               \\\midrule
        Total params:                    & \multicolumn{2}{r}{60286470 (299.97 MB)}                     \\
        Trainable params:                & \multicolumn{2}{r}{286470 (1.09 MB)}                         \\
        Non-trainable params:            & \multicolumn{2}{r}{60000000 (228.88 MB)}                     \\
        \bottomrule
    \end{tabular}
\end{table}

Đối với dữ liệu train, có thể thấy với mỗi epoch, lượng mất mát ngày càng giảm và tới cuối cùng gần như không còn giảm quá đáng kể. Cùng với đó là sự đi lên của các thông số đánh giá. Ban đầu Accuracy của mô hình quá cao có thể do quá khớp, nhưng dần dần đã giảm xuống ở mức ổn định hơn. Dựa trên kết quả từ các epoch và biểu đồ bên trên, có thể thấy rằng mô hình có thể cho ra độ chính xác tương đối cao, trên $90\%$ và giá trị mất mát cũng rất khả quan.

Đối với tập dữ liệu valid, các thông số tuy có nhiều sự thay đổi trong quá trình huấn
luyện, nhưng không chênh lệch quá nhiều so với kết quả của tập train. Từ đó có thể kết
luận rằng mô hình có thể hoạt động tương đối ổn.
Đánh giá mô hình dựa trên tập dữ liệu test.

Đánh giá dựa trên tập test cho kết quả tương đối khả quan dù không cao như tập chính,kết quả vẫn lên tới $95\%$. Điều này cho thấy rằng mô hình được huấn luyện cho kết quả tương đối tốt. Độ chính xác của mô hình có thể lên tới $95\%$, kết quả này cho thấy mô hình có thể dự đoán tốt hầu hết kết quả nhưng cũng không bị overfitting.


\subsubsection{Mô hình GRU}
Với mô hình GRU có epochs = 32, batch = 128, mô hình gồm 4 lớp (bảng \ref{table:gru-model}):
\begin{itemize}
    \item 1 lớp embedding sử dụng fasttext embedding
    \item 2 lớp Bidirectional LSTM
    \item 1 lớp dense
\end{itemize}

\begin{table}[htb]
    \centering
    \caption{Bảng tóm tắt mô hình GRU}\label{table:gru-model}
    \begin{tabular}{l l r}
        \toprule
        \textbf{Layer (type)}            & \textbf{Output Shape}                    & \textbf{Param \#} \\\midrule
        input (InputLayer)               & [(None, 200)]                            & 0                 \\
        embedding (Embedding)            & (None, 200, 300)                         & 60000000          \\
        bidirectional (Bidirectional)    & (None, 200, 128)                         & 140544            \\
        dropout (Dropout)                & (None, 200, 128)                         & 0                 \\
        bidirectional\_1 (Bidirectional) & (None, 128)                              & 74496             \\
        dense (Dense)                    & (None, 6)                                & 774               \\\midrule
        Total params:                    & \multicolumn{2}{r}{60215814 (229.71 MB)}                     \\
        Trainable params:                & \multicolumn{2}{r}{215814 (843.02 KB)}                       \\
        Non-trainable params:            & \multicolumn{2}{r}{60000000 (228.88 MB)}                     \\
        \bottomrule
    \end{tabular}
\end{table}

Đối với dữ liệu train, có thể thấy với mỗi epoch, lượng mất mát ngày càng giảm và tới cuối cùng gần như không còn giảm quá đáng kể. Cùng với đó là sự đi lên của các thông số đánh giá. Ban đầu Accuracy của mô hình quá cao có thể do quá khớp, nhưng dần dần đã giảm xuống ở mức ổn định hơn. Dựa trên kết quả từ các epoch và biểu đồ bên trên, có thể thấy rằng mô hình có thể cho ra độ chính xác tương đối cao, trên $90\%$ và giá trị mất mát cũng rất khả quan.

Đối với tập dữ liệu valid, các thông số tuy có nhiều sự thay đổi trong quá trình huấn
luyện, nhưng không chênh lệch quá nhiều so với kết quả của tập train. Từ đó có thể kết
luận rằng mô hình có thể hoạt động tương đối ổn.

Đánh giá dựa trên tập test cho kết quả tương đối khả quan dù không cao như kết quả huấn luyện, độ chính xác Accuracy đạt trên $90\%$. Điều này cho thấy rằng mô hình được huấn luyện cho kết quả tương đối tốt. Độ chính xác của mô hình có thể lên tới $91\%$, kết quả này cho thấy mô hình có thể dự đoán tốt hầu hết kết quả nhưng cũng không bị quá khớp. Tuy nhiên so với mô hình LSTM trước đó thì kết quả này có phần lép vế.

\subsubsection{Tinh chỉnh mô hình PhoBERT}

\subsubsection{Đánh giá giữa 3 mô hình}

\begin{table}[htb]
    \centering
    \caption{asd}
    \begin{tabular}{l r c c c}
        \toprule
        \textbf{Model} & \textbf{Param \#} & \textbf{Precision} & \textbf{Recall} & \textbf{Accuracy} \\\midrule
        2-layer biLSTM & 60M               & ?                  & ?               & ?                 \\
        2-layer biGRU  & 60M               & ?                  & ?               & ?                 \\
        PhoBERT-base   & 135M              & ?                  & ?               & ?                 \\
        \bottomrule
    \end{tabular}
\end{table}