\section*{Thành tựu}
Về mặt lý thuyết: Tìm hiểu về các mô hình mạng thần kinh trong học sâu (CNN, RNN, LSTM, GRU) và các kỹ thuật xử lý văn bản ngôn ngữ tự nhiên (Word2vec, GloVe, Fasttext).\\
Về mặt ứng dụng: Xây dựng hai mô hình mạng thần kinh LSTM và GRU kết hợp kỹ thuật nhúng từ Fasttext để phát hiện văn bản độc hại.

\section*{Hạn chế}
Dữ liệu được sử dụng là dữ liệu tiếng Anh được dịch lại tiếng Việt, do đó có những hạn chế về mặt huấn luyện do dữ liệu có thể thiếu chính xác.\\
Mô hình dự đoán tương đối ổn nhưng đôi khi không thể nhận diện được chính xác mức độ độc hại của văn bản.

\section*{Hướng phát triển}
Do dữ liệu được dịch từ tiếng Anh qua tiếng Việt, nên đôi chỗ còn khó hiểu và không được chính xác về mặt ngữ pháp; vậy nên cần phải xử lý và dịch lại tập dữ liệu, đồng thời thêm vào dữ liệu thuần Việt để tăng độ chính xác.\\
Đối với mô hình LSTM và GRU còn cho ra một số kết quả chưa được chính xác cần phải cải thiện thêm. Có thể tìm hiểu thêm một số mô hình hiệu quả hơn trong xử lý ngôn ngữ tự nhiên.\\
Nhóm cũng hướng tới việc ứng dụng mô hình vào trong môi trường thực tế như: ứng dụng nhắn tin, kiểm tra văn bản, \dots